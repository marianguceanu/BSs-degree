\chapter{Introduction}

%\chapter*{Introducere}
\label{intro}


Cross-platform development is a modern approach to transitive applications.
It is a domain that gained popularity along the years, due to constant innovations in the gadget realm, such as the smartphone, the tablet or the game console.
Each of these took a different approach on their system design or their operating system, hence creating a multitude of "platforms", all with unique features and perspectives regarding development.

\par
This thesis will have \textit{4} main chapters.
In the next chapter, the second one, I will take a look through the history of cross-platfrom development and the principle of separation of concerns.
This is a necessary step in order to understand the current state of this domain, and see how separation of concerns can be applied to it.

\par
In the third chapter, I will take a look at the principle of separation of concerns.
This is a development principle that has been around for a long time, and it is a key factor in the development of software.
A thorough analysis over the positive and negative outcomes that separation of concerns can have will be made.
Then I will make a point by point cover over the advantages this principle proposes and see how it aligns with the upgrades that this branch needs.
This will be done through analyzing all the conditions that it proposes for a concern to meet in order to be separated.


\par
The fourth chapter will cover the branch of cross-platform development.
I will take a look at the current state of this domain, and see how separation of concerns can be applied to it.
This will be done through analyzing all the design principles that are actively used and see how they can be separated into different concerns.
One key feature of the application of this principle will be design patterns.
Yet another aspect that we will take a look at and see the ones used in the proof of concept application.
Last but not least, we'll take a look at all the tools that are used in the development of cross-platform applications.

\par
The fifth chapter will cover the implementation of a proof of concept application.
This will be a simple application that will be developed using the principles of separation of concerns.
The application will be developed using the tools that are currently used in the development of cross-platform applications.
It will follow the principles of separation of concerns, and also the aforementioned design patterns.
Positive, negative and neutral outcomes will be analyzed, and the results will be presented.


\par
The final chapter will cover the conclusions of this thesis.
I will take a look at the results of the proof of concept application, and see how separation of concerns can be applied to cross-platform development.
I will also take a look at the future of cross-platform development, and see how separation of concerns can be used to improve it.
Finally, I will present the conclusions of this thesis, and see how separation of concerns can be used to improve cross-platform development.
