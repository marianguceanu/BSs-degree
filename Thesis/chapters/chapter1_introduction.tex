\chapter{Introduction}

%\chapter*{Introducere}
\label{intro}


Cross-platform development is a modern approach to transitive applications.
It is a domain that gained popularity along the years, due to constant innovations in the gadget realm, such as the smartphone, the tablet or the game console.
Each of these took a different approach on their system design or their operating system, hence creating a multitude of "platforms", all with unique features and perspectives regarding development.
\par
This thesis will have \textit{n} main chapters.
In the first one, we will take a look at the history of development as a whole, and see the turns it took along the years.
In this chapter I will also analyze the evolution of computers, from the earliest times until present.
Then I will link this evolution to software development, to see why cross-platform development turned from a very specific niche to an actual problem.
\par
In the second chapter, I will analyze the current state of cross-platform development.
The first subject will cover the lack of separating concerns in this branch.
I will analyze statistics and observe the consistency that has been preserved throughout the years.
Then, the second subject will be what this branch kept as legacy from its parent derivatives.
To be more exact, this will also be divided into three areas of concern: what are the positive features that live on, regarding several aspects such as performance and code maintenance, what are the negative features that continued out of pure legacy and comfort, and also the features that lie in between the ones aforementioned.
\par
Third chapter will contain an overview and a history of separating concerns as a development principle, and also from an architectural point of view.
This will be a thorough analysis over the positive and negative outcomes that separation of concerns can have.
Then I will make a point by point cover over the advantages this principle proposes and see how it aligns with the upgrades that this branch needs.
\par
In the fourth chapter I will go into more details about the separation of concerns principle.
There will be an in-depth approach towards how it is also separated into different notions, layers and practices.
I will then take the already in-use principles of cross-platform development and separate them, so that they will be fitting the aforementioned layers.
\par
Fifth chapter will cover the impacts of separating concerns.
Here I will do a thorough analysis on the theoretical advantages and disadvantages, and then proceed to measure them.
Collected data will sum up to a report that should prove the above points, all within a given metric.
I will analyze code maintainability, reusability, scalability and more, and see the time complexities, measure memory usage, and then present the challenges regarding debugging, tooling and the learning curve.
