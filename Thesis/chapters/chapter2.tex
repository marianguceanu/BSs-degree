\chapter{History}
\label{chap:ch1}

\indent

\par
In this chapter, I will elaborate on the progresses made throughout the decades regarding designing software and the evolution of hardware, 
and see how we ended up with cross-platform development as a necessity.
I will also discuss the first mentions of the term "cross-platform development" and the first toolkits that were created to help developers in this area.
Another main aspect of this chapter is the principle of separation of concerns.
I'll take a look at the history of this principle and how it was implemented in the early days of software development.

\section{History of cross-platform development}
In 1972, Dennis M.
Ritchie and his team created the C programming language.
To say it was revolutionary, was an understatement.
This project turned Dennis Ritchie into a titan and unforgettable figure of the software world.
He was a "bearded, somewhat disheveled computer scientist"\cite{ritchieJobs} that stays at the foundation of the things we make most use of today, for exmaple the internet.
\begin{figure}[htbp]
    \centering
    \includegraphics[scale=0.7]{pictures/KenThompson_and_DennisRitchie1973.jpg}
    \caption{Ken Thompson and Dennis Ritchie, 1973}
    \label{unixMakers}
\end{figure}

\par
C is a language that derived from BCPL (Basic Combined Programming Language).
It became relevant when, yet again, Dennis Ritchie decided to use it in order to re-write the source code of the UNIX operating system.
Such and operation was deemed necessary by him mainly because the U.S government categorized the capitalism of such a product as a violation the "Sherman Antitrust Act"\cite{unixRepo}.
Due to this, AT\&T released the Operating System as royalty-free, but later however licenses became a bigger obstruction, hence the limitations in the access of the source code.
Then Dennis Ritchie decided to re-write the whole OS in the high-level programming language of C, in 1973.
This project, although later forked into many other operating systems such MacOS or GNU/Linux, was maintained and developed up to this day to the FreeBSD project.

A part of the legacy assembly source code of UNIX is still on GitHub, at the repository 'dspinellis/unix-history-make'.
\begin{figure}[htbp]
    \centering
    \includegraphics[scale=0.5]{pictures/unix_repo.png}
    \caption{Assembly source code of UNIX on GitHub}
    \label{unixRepo}
\end{figure}
During the 80's, software design became an area on interest for most people involved in the computer science field.
One of the earliest approaches of software design is the Layered architecture.
This was firstly introduced as a need because of the way computer networks were operating.
Now, files were getting bigger and because of that, "The programmer must maintain descriptions of
how various alternative designs are distributed among files"\cite{layeredArchitecure80s}.

\par

Along with these software design paradigms, this decade introduced the first implementation of a version control system.
It worked quite similarly to the modern approaches.
It stored the differentials of the files in a 'delta' file\cite{layeredArchitecure80s}.
This file was one of the deterministic factors for which version control and layered architecture were introduced.

As files grew bigger in size, the delta file associated with it was also getting clunkier, thus making reverting or alternating changes a time and memory-consuming task, and by separating the files into smaller ones, each depending on the other one, this process would be optimized.

\par
This decade brings the most enhancing and used tool of the modern world, the internet.
Although it is believed that invented rather earlier, in the late 60's, it is in 1983 when ARPANET and DDN (Defense Data Network) switched to the TCP/IP standard, "so that computers of many kinds could communicate with each other, no matter what kind of network they were connected to in the Internet".\cite{arpanetDdn} This was a big leap, as ARPANET was an already enormous cluster of interconnected computers, therefore the computers entering the network would have access to them.
\begin{figure}[htbp]
    \centering
    \includegraphics[angle=270,scale=0.35]{pictures/arpanet.png}
    \caption{ARPANET logical map, 1983}
    \label{arpanetLogicaMap}
\end{figure}

\par
Due to this, a doubling of household computers happened until the 90's.
This meant that more people were put on the network, and, due to this, all the computer owners had access to each other's data and information, should they wanted to access it.
\begin{center}
    \begin{tabular}{|| c | c ||}
          \hline
          \textbf{Year} & \textbf{Percentage} \\
          \hline
          1980 & 2.8\% \\
          \hline
          1982 & 6.2\% \\
          \hline
          1983 & 7.3\% \\
          \hline
          1984 & 8.4\% \\
          \hline
          1986 & 9.5\% \\
          \hline
          1988 & 12.5\% \\
          \hline
          1989 & 15.0\% \\
          \hline
    \end{tabular}
\end{center}

\par
As you can see, the biggest jump in the percentages happened between 1980 and 1982.
But, since ARPANET switched their protocols in 1983, the number of computers in personal households has grown by a staggering 7.7\%, a bit more than double.
\par
In 1990, we see the first true creation of cross-platform software.
At CERN, Switzerland, Tim Berners-Lee submits two ideas for what should be the Web, back in March 1989.
Until the Christmas of 1990, he contoured the World Wide Web, making the first prototype for it.
The intention of this project was to be "a pool of human knowledge", thus allowing for ease of communication between people who were geologically disjointed\cite{worldWideWeb}.

\par
This project offered features that are still generally available today, such as a server, HTML, the first web programming language, URL's and the first browser.

\subsection{First definition}
Now, during this decade we can see some of the first mentions of the term "cross-platform development".
That is due to the fact that more and more "platforms" (example: Windows, MacOS) started rising and gaining popularity, and certain large applications needed to behave and act accordingly on all of them.
I will refer to "platform" from now on as development environments, for example Android meaning not only the mobile phone operating system, but also the ecosystem surrounding it, such as smartwatches or TV's.

\par
Some of the first people that proposed this terminology were Bishop J.
and Horspool N., who also gave the first definition to this development branch.
They described the product of it as "software that exists in different versions so that it is available on more than one platform"\cite{firstDefinition}, all the way back in 2006.
In this definition, their meaning of platform is the same as the one I mentioned earlier, because back then the definition was a bit more general, such as "language, operating system, computer, or some combination".
In the same time-period, the Web was starting a world-conquering journey.
Along with it, it brought a whole toolkit of development principles and design patterns, with component-based development being the most praised one.
This is also mentioned and observed by the aforementioned authors, whom agree that component-based development  has made a good impact  toward satisfying "both parts of the definition."\cite{firstDefinition}

\subsection{First toolkits}
The idea of cross-platform development was initially brought as a fix for one of the bigger issues developers faced at the time, and that is building graphical user interfaces (GUI's).
This in itself represented a dilemma because for every platform there were dedicated tools and specific programming languages, for example for Windows there was exclusively .NET with the C\# programming language.
This inherently meant that there was no uniform way of developing such GUI's across platforms.
There is also acknowledgement in this, starting with the earliest days.
Due to such history, there came the stigma that cross-platform toolkits are "integral to GUI development"\cite{firstDefinition}.

\par
Because of this, we can also observe the rise in popularity of some toolkits that were not very seriously taken considered before this realization, such as Qt or GTK+.
These are some of the first cross-platform frameworks that are still used today at a production level, and we can see them used in many well-known applications, such as Spotify for Qt, or the GNOME desktop environment for GTK+.
They are trying to achieve different goals, but one thing that they have in common is that both Qt and GTK+ define a collection of widgets that should work as "platform independent"\cite{firstDefinition}.
This then allowed for the developers to apply a "learn once, code everywhere" approach.

\section{History of the principle of separation of concerns}
The principle of separation of concerns has roots that extend back several decades, and it has been discussed and applied in various forms across different disciplines, including software engineering, information architecture, and systems design.
While it's challenging to pinpoint an exact date or origin for this principle, we can explore some key moments in its history, the ones that got it started as a founding principle in software design.
First we'll take a look at the structured programming era, this taking place during the 1960's-1970's timeframe.
\par
During the structured programming era, pioneers such as Edsger W.
Dijkstra and Niklaus Wirth advocated for modular programming techniques that emphasized separation of concerns to improve code clarity and maintainability.
Dijkstra was the first to indirectly point towards this approach, because he argued against the unstructured use of goto statements and advocated for structured programming techniques, which inherently promote separation of concerns.
He emphasized the importance of clear control flow and modular design in creating understandable and maintainable code.
He believed that a so called "coordinate system"\cite{firstDefinitionPrinciple} can be used to describe the code in a "helpful and manageable"\cite{firstDefinitionPrinciple} manner.

