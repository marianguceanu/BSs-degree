\chapter{Conclusions}

The research has shown that the use of a cross-platform framework can be beneficial for companies that want to develop applications for multiple platforms.
It can reduce the development time, as the code can be shared between the platforms.
This can also reduce the costs of development, as the developers don't have to learn multiple languages and frameworks.
The research has also shown that the use of a self-implemented server can be beneficial for companies that want to have more control over their data.
It can provide a more realistic scenario, where the data is stored in a company's database, rather than in a third party service.

\par
As noticed in the examples above, the server is separated into different layers, each with its own responsibilities.
The frontend is also separated into different layers, but it requires some manual tweaking in certain use cases.
Both have proven that this separation of concerns can make the code more readable and maintainable, through good and bad examples.
This separation of concerns can also make it easier to add new features to the application, as the code is more organized.

\par
Through the practical implementation of the application, I have learned that the use of a cross-platform framework can be beneficial for companies that want to develop applications for multiple platforms.
Not only that, but the reduction in code duplication can also reduce the development time and costs.
By making use of thoughtful separation of concerns, the code can be more readable and maintainable, and it can be easier to add new features to the application.

\par
It is important to note that the research has some limitations.
The application is developed for Android and Linux devices only, and it uses a self-implemented server.
This means that the application is not available for iOS devices, and it does not use a third party service such as Firebase.
This can limit the reach of the application, as it is not available for all devices.
It can also limit the scalability of the application, as the server is not as robust as a third party service.
But even with these limitations considered, we can still see the benefits of using a cross-platform framework and a self-implemented server, with separation of concerns.


\label{conclusions}
