\chapter{Current state of cross-platform development}
\section{Overview}
Cross-platform software is designed to operate on multiple computing platforms, such as different operating systems (Windows, macOS, Linux) or device types (desktop computers, mobile devices, web browsers).
The primary goal of cross-platform development is to write a single codebase that can run on various platforms with minimal modifications, thereby reducing development time and cost while maximizing reach and usability.
Now we will take a look at some key-features of this type of development.
\subsection{Code Reusability}
This development area can be best categorized into two two types of codebases: single codebase and shared libraries and frameworks.
\par
Developing with a single codebase means that the same code can be reused across multiple platforms, reducing duplication and making maintenance easier.
Also, due to this approach, the need for dependency checking is eliminated, as it is one of the most daunting tasks regarding cross-platform development.
A good scenario, for example, is to know that your application's server is dependent on an Object-Relational Mapping (ORM) framework, or a wrapper for one to be more specific, and to find that it is being discontinued.
This is where single codebase comes in, as it is not dependent  


\section{Lack of separartion of concerns}
\section{Legacy features that are present today}
\section{Tools and techniques}
\label{chap:ch4}

